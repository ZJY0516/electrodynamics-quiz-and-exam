\documentclass{exam}

\usepackage{ctex}
\usepackage{amsmath}
\usepackage{physics}
\usepackage{anyfontsize}

\renewcommand{\thequestion}{\zhnum{question}}
\renewcommand{\questionlabel}{\thequestion .}
\renewcommand{\thepartno}{\arabic{partno}}
\renewcommand{\partlabel}{\thepartno .}

\author{朱江云(整理)}
\title{电动力学第二章测验}
\date{}

\begin{document}
\maketitle
\begin{questions}
    \question 推导电磁场动量守恒定律,写出电磁场动量密度和动量流密度张量的表达式,简述平面电磁波的动量流密度张量的形式。
    \question
    \begin{parts}
        \part 当两种绝缘介质的交界面上没有自由电荷时,试证明交界面两侧电力线与交界面法线的夹角$\theta_1$和$\theta_2$满足
        \begin{equation*}
            \frac{\tan \theta_1}{\tan \theta_2}=\frac{\varepsilon_{1r}}{\varepsilon_{2r}}
        \end{equation*}
        式中$\varepsilon_{1r}$和$\varepsilon_{2r}$分别是两介质的介电常量。
        \part 当两种导电介质内都有恒定电流时,试证明交界面两侧电力线与交界面法线的夹角满足
        \begin{equation*}
            \frac{\tan \theta_1}{\tan \theta_2}=\frac{\sigma_{1r}}{\sigma_{2r}}
        \end{equation*}
        \part 当导体(电导率为$\sigma$)和绝缘体(电容率为$\varepsilon$)接触时,试求交界面两侧电力线与法线的夹角大小。
    \end{parts}
    \question 内外电极的截面半径分别为a和b的无限长圆柱形电容器,单位长度荷电为$\lambda_f$,两
    极间填充电导率$\sigma$的非磁性物质。
    \begin{parts}
        \part 证明在介质中的任何一点传导电流与位移电流严格抵消,因此内部无磁场;
        \part 求$\lambda$随时间衰减的规律;
        \part 求与轴相距为$r$的地方的能量耗散功率密度;
        \part 求长度为$l$的一段介质总的能量耗散功率,并证明它等于这段的电场能量减少率。
    \end{parts}
\end{questions}
\end{document}