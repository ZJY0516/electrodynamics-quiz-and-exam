\documentclass{exam}

\usepackage{ctex}
\usepackage{amsmath}
\usepackage{physics}
\usepackage{anyfontsize}
\usepackage{bm}
\usepackage{tikz}
\usepackage{esvect}

\renewcommand{\thequestion}{\zhnum{question}}
\renewcommand{\questionlabel}{\thequestion .}
\renewcommand{\thepartno}{\arabic{partno}}
\renewcommand{\partlabel}{\thepartno .}

\author{朱江云(整理)}
\title{电动力学第六章测验}
\date{}

\begin{document}
\maketitle

\begin{questions}
    \question
    \begin{parts}
        \part 写出时谐电流推迟势的多级展开式及其零级和一级展开的物
        理意义
        \part 分别写出磁偶极子和电四极辐射对应的磁场、电场、平均能流密
        度以及总辐射功率公式
    \end{parts}
    \question 一电偶极子的电偶极矩$\vv{p}=\vv{p}_0 \cos{\omega t}$,
    $\vv{p}$为常矢量。以$\vv{p}$所处为原点$O$,取球坐标如图所示
    \begin{parts}
        \part 试由$\vv{p}$的推迟势求远区(即$r\gg \lambda=\frac{2 \pi c}{\omega}$)的矢势$\vv{A}$
        \part 由$\vv{A}$求辐射场
        \part 求辐射的平均能流密度和总功率
    \end{parts}
    \question 如图所示,$xOy$平面中心的线段(长为$2a$)以恒定角速度$\frac{\omega}{2}$绕$z$轴旋转,线段两端各有一个电量为$e$的点电荷。
    \begin{parts}
        \part 计算体系的电、磁偶极矩及电四极矩
        \part 说明远场辐射类型与辐射频率
        \part 在远场$P(R,\theta,\phi)$处观察辐射,则$\theta=0°$和$\theta=90°$时偏振状况分别如何
        \part 将其中一个点电荷的电量替换为$-e$,求体系的远场辐射总功率
    \end{parts}
\end{questions}
\end{document}